% !TeX root = main.tex

\section{Примеры \LaTeX}

\subsection{Пробелы и чёрточки}
Тире "--- такая штука, которая в русском немного не такая как у всех.
Для прямой речи "--*. Для Имён "--~ Составных
Для И.~Н.~Ициалов есть \href{https://ru.wikibooks.org/wiki/LaTeX/%D0%A4%D0%BE%D1%80%D0%BC%D0%B0%D1%82%D0%B8%D1%80%D0%BE%D0%B2%D0%B0%D0%BD%D0%B8%D0%B5_%D1%82%D0%B5%D0%BA%D1%81%D1%82%D0%B0#%D0%9D%D0%B5%D1%80%D0%B0%D0%B7%D1%80%D1%8B%D0%B2%D0%BD%D1%8B%D0%B9_%D0%BF%D1%80%D0%BE%D0%B1%D0%B5%D0%BB}{неразрывный пробел}.

\subsection{Единицы измерения}
\verb!\qty{10}{\kilo\hertz}! "--- \qty{10}{\kilo\hertz}

\verb!\qty{2000}{\pieces}! "--- \qty{2000}{\pieces}

Углы в градусах можно короче \verb!\ang{45}! "--- \ang{45}

Числа для правильного форматирования \verb!\num{1.7e3}! "--- \num{1.7e3}

Можно вывести список чисел \verb!\numlist{1.7e3;4.5;123}! "--- \numlist{1.7e3;4.5;123}.

Или величин \verb!\qtylist{1.7e3;4.5;123}{\kilo\hertz}! "--- \qtylist{1.7e3;4.5;123}{\kilo\hertz}

\subsection{Шрифты}
\textmd{Hello. Привет.}

\textsf{Hello. Привет.}

\textrm{Hello. Привет.}

\texttt{Hello. Привет.}

\textbf{Hello. Привет.}

\textit{Hello. Привет.}

\textsl{Hello. Привет.}

\textsc{Hello. Привет.}
